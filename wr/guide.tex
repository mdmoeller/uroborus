\documentclass[english]{article}

\usepackage[latin9]{inputenc} 
\usepackage[letterpaper]{geometry}
\geometry{verbose,tmargin=1in,bmargin=1in,lmargin=1in,rmargin=1in} 
\usepackage{amsmath}
\usepackage{amssymb} 
\usepackage{cite}
\usepackage{xcolor}
\usepackage{colortbl}


% Typeset UROBORUS:
\newcommand{\Uro}{\textsc{Uroborus}} \newcommand{\Taran}{\textsc{Tarantula}}

% The `example' files referred to by this text:
\newcommand{\fc}{\textbf{faultycode}} 
\newcommand{\fcp}{\textbf{faultycode.py}}
\newcommand{\fcT}{\textbf{faultycodeTest.py}}
\newcommand{\fcip}{\textbf{faultycode\_instrumented.py}}
\newcommand{\fcc}{\textbf{faultycode\_coverage.txt}}
\newcommand{\fcpf}{\textbf{faultycode\_passfail.txt}}
\newcommand{\fch}{\textbf{faultycode\_report.html}}
\newcommand{\RO}{\texttt{RuntimeOracle}}




\title{Fault Localization with Uroborus} \author{Samarth Kishore \and Daniel Klein \and Mark Moeller
\and Twisha Shah \and \{samarthk, kleindan, mmoeller, twisha\}@seas.upenn.edu}



\begin{document} \maketitle


\section{Introduction}

\Uro\ is a fault localization tool for Python built in the style of Jones, Harrold and Stasko's
\Taran ~\cite{Jones}.

Suppose we have written a Python module \fcp\ that we we want to debug using a visualization
strategy like \Taran. We have developed a tool, \Uro, which does exactly this. 
To use \Uro\ to debug a Python module, we need to do the following: 
\begin{enumerate} 
\item We write a set of \Uro\ tests for
\fc\ that call the methods of the module we want to test.  
\item We run the \Uro\ on the test
module, which generates datafiles detailing statement coverage and assertion passes and failures.
\item \Uro\ finally generates a ``report,'' which is an HTML file representing the source code, colored
according to the pass/failure calculations in the style of \Taran. We use this report to help
localize where the faulty lines likely are. 
\end{enumerate} 
\textbf{Note:}
At this time, \Uro\ is only able to test the \emph{methods} of an input source file, a la unit
testing. That is, executable
scripts are not handled under the current framework, but are an area being considered for further
improvement of the tool.

\section{Using Uroborus} \subsection{Creating \Uro\ Test Modules}

A developer first needs to create \Uro-style tests for \fc. To do that, we just have to write a
module, say \fcT, that has the following header:\\ 

\begin{verbatim} 
#!/usr/bin/python 

import urotest
faultycode = urotest.uro_import('faultycode') 
\end{verbatim}

From there, the test suite consists of \emph{methods} which are tests of the desired methods of \fc.
That is, the script need not do anything when run directly. In particular, the test module
\emph{should not} call any methods of instrumented code if run directly.  Each method of the test
suite should take one parameter, and that parameter is the \RO\ that will be used
for that test case. For example, we might have:

\begin{verbatim}
def test_that_add_returns_six(R):
    value = faultycode.add(2, 4)
    R.assertEquals(6, R)
\end{verbatim}

The developer may optionally include a method called \texttt{init} if some intial set up is
required. \Uro\ will run \texttt{init} if it exists before any other tests in the module.

Note that \emph{any} methods at the package level in \fcT\ will be assumed to be an \Uro\ test that
takes a \RO\ as an argument.  If a helper method is required for one or more test methods,
it should be encapsulated within a class in \fcT\ so that it is not run directly as an \Uro\ test.

\subsection{Localizing Faults}

Next, the developer runs the tests. This is done with:\\
 
\texttt{\$ uroborus }\fcT \\

The module-level methods will each be run in turn by \Uro, very similarly to how JUnit runs a set of
methods. Order of execution of these methods is not defined, except that \texttt{init} will be
called first.

All assertions must be done with \RO's assertion methods (for details, see the
details regarding \RO\ in section \ref[ro]).

When the tests are run, \Uro\ will generate \fcc\ and \fcpf, and it will write information to
standard out about the tests that pass and fail. 

Finally, \Uro\ generates the HTML report. It will output a file called \fch. This file is the
fault-visualized version of the source code.


\section{Implementation}

The following sections provide details regarding the implementation decisions of \Uro.

\subsection{Overview}

A single run of \Uro\ is divided into three stages: Instrumentation, Test Running, and Report
Generation. The source file being tested and the test suite are the input to Instrumentation. After
that, the output of each stage is the input to the next one, ultimately concluding with the
generation of the final visualization.

\subsection{Instrumentation}

The module \textbf{instrument.py} is one of the core components of \Uro. Its function is to
``instrument''
the source file by inserting a command to write to a data file before nearly every line. 
It instruments only those lines of code that can get executed at
run time. The instrumented code is then used in place of the original code for user generated tests.
The file writes that get instrumented with the source lines enable \Uro\ to keep track of the lines
that get executed during different runs/tests. In the test running stage, a coverage file is
generated as a result of the file writes in the instrumented file that keeps track of the lines that
were executed in each run.

The challenge with instrumenting source files is when there are classes. The \Uro\ instrumenter
takes care not to instrument class and function defs, as those would be executed on import. 
Since there is a small setup required that must happen after the source is imported before testing
can begin, the importation itself cannot start logging statement executions.
Since python uses indentation to delimit
blocks, the file writes are indented at the same level as
the line being instrumented. Also, statements like \texttt{else} and \texttt{catch} are not
instrumented, because they cannot be syntactically captured by the insertion of a preceding line.

Another feature of the \textbf{instrument.py} module is the way it handles line continuation in Python. Since
Python allows a single statement to span several lines in some cases, the instrumenter is careful not to instrument
in the middle of a statement. This was somewhat tricky to implement, because a line break is the
normal statement delimiter in Python.

In the typical use case, \Uro\ calls \textbf{instrument} on-the-fly. However, some source code can
be instrumented independently by running:\\
\begin{verbatim}
$ instrument.py faultycode.py
\end{verbatim}

\subsection{Running Uroborus Tests} \label[ro]

Generating a visualization like \Taran's presents a difficult challenge: it combines the need for
simple statement coverage information with information about passing and failing tests in an
interconnected way. Normally the two sets of information, coveraging and passrate, are orthogonal
issues and can be handled by separate utilities. For example, in the Java environment, Cobertura
handles coverage, while JUnit handles passing and failing of tests. 
\Uro\ handles their combination with 
a class called \RO. A single \RO\ is constructed for the entire 
the test suite (this is not really a Singleton because it is conceivable that multiple
\RO's might be needed if \Uro\ was expanded to support multiple-file source modules).

A reference to the \RO\ is coded into the instrumented source file, as well as passed to
each test method as a parameter. The \RO\ serves several purposes. First, it numbers each
test method called as ``runs,'' and provides the instrumented source code with access to what run is
being executed at a particular time. That way, the instrumented code can output to the coverage file
which statements are executed by each run. Next, the \RO\ provides the assertion checking
that is necessary to writing unit tests. Each successive assertion called on the \RO\ is
implicitly `and'-ed until the \RO\ is notified that the run is complete (either the test
method completed or an exception was raised). At that point, if the result of the `and' of all of
the assertions is recorded as the result of that run.

\RO\ provides the following methods:
\begin{enumerate}
\item \textbf{getRunNum(self)} \\
    Returns the number of the current run. This is not meant to be used by the test developer in
    tests, but rather by the instrumented code only.
\item \textbf{assertTrue(self, expr)} \\
    This method performs the most basic assertion check. Passing a False (typically the result of an
    expression) to \texttt{assertTrue} will result in a failure for the run.
\item \textbf{assertEquals(self, arg1, arg2)} \\
    Equivalent to \texttt{assertTrue(arg1 == arg2)}. 
\item \textbf{run\_complete(self)} \\
    Notify this \RO\ that a test has completed and that the passing or failing status of
    the current run should be recorded, and the next run should be initiated.
\item \textbf{except\_fail(self)} \\
    Notify this \RO\ that the run ended because of an exception, and should be immediately
    failed.
\item \textbf{passes(self)}  \\
    Returns the number of passes which have been completed.
\item \textbf{fails(self)}  \\
    Returns the number of failures which have been completed.
\item \textbf{tests\_complete(self)} \\
    Notifies the \RO\ that all tests have been completed so that it can close its open
    filestreams.

\end{enumerate}

In all, there are two data files output by the test running stage. The coverage file denotes which
statements were executed in each run.
This information appears in the file as a column of statement numbers separated by a tab from a
column of run numbers.
The pass/fail file, on the other hand, denotes which runs passed and failed.
This information appears as a column of run numbers separated by a tab from a column of 1's and 0's,
with 1's representing a passing result.
By combining the information of these two files, the contribution of a single statement towards
passing or failing can be analyzed.

\subsection{Report Generation}

\begin{enumerate}
\item \textbf{DisplayResult.java} \\
This class takes the module name as an argument. It creates an instance of the class Computation and calls its methods.
In a typical run of \Uro, the call to DisplayResult is made automatically. However, if for some
reason one needs to generate the report independently, one should run:
\begin{verbatim}
$ java DisplayResult faultycode
\end{verbatim}
Display result infers the names of the required data files, and looks for them in the current
directory.

\item \textbf{Computation.java} \\
This class recognizes three files using filename as \fcp, \fcpf, and
\fcc. The explanation of its methods are as follows.

 \begin{enumerate}
 \item \textbf{codeToString()} \\
This method reads the \fcp\ file line by line and stores it in a String array code. While
adding each line to \texttt{code[]}, a statement number (starting from 1) is appended at the beginning of
each line and stored in \texttt{code[]} such that every line can be indexed in \texttt{code[]} using its statement
number. \\
e.g. Statement 5 is stored in \texttt{code[5]}.

\item \textbf{getCount()} \\
This method simply counts the number of lines in the \fcpf\ file.

\item \textbf{pfArray()} \\
A passfail array is declared with length equal to the number of lines in
\fcpf\ file.
For every line in the \fcpf\ file, the first numeric value (run number) is stored in
a variable \texttt{index} and the other numeric value (0 or 1, which represents fail or pass) is stored in
\texttt{passfail[]} at the location stored in index.
e.g. If run number 2 has passed, \texttt{passfail[2]=1}.

\item \textbf{pfCal()} \\
This method calculates the total number of runs that passed and the total number of runs that
failed. It runs a for loop through each element of \texttt{passfail[]} and increments \texttt{pass} if the value
stored is 1 and increments \texttt{fail} if the value is 0.

\item \textbf{statementPassFail()} \\
This method calculates the number of times each statement of the code passed as well as the number of times it failed.
It first creates an array \texttt{temp\_array} of size equal to the number of lines in
\fcc.
file. A two dimensional array \texttt{statement\_passfail[][]} is created to store the number of passes and fails
for each statement. Each line in the \fcc\ file is read and checks if that same line
    exists in the \texttt{temp\_array[]}. If it exists, it does nothing and reads the next line. If it does
    not exist, the line is added in the next available location in \texttt{temp\_array[]}. Using space is
    delimiter the first and second numeric values in the line are stored in the variables
    \texttt{statement\_no} and \texttt{run\_number} respectively. 
    To check whether this \texttt{run\_number} had passed or failed,
    it checks the value of \texttt{passfail[run\_number]} which would be 1 or 0 respectively. If it is 1,
    \texttt{statement\_passfail[statement\_no][1]} is incremented by 1 otherwise
    \texttt{statement\_passfail[statement\_no][0]} is incremented by 1.

\item \textbf{getColorValue()} \\
This method calculates the color value of each statement of the code ranging between 0 and 120.
For each statement, its passing percentage (\texttt{passed}) is calculated as number of runs in which it
passes divided by the total number of passing runs. Its failing percentage (\texttt{pfailed}) is calculated
similarly. These values are stored in a results array. The color value is calculated as
(passed/(passed+pfailed))*120.

\item \textbf{createOutput()} \\
        The \texttt{create()} method of CreateHTML class is called.
\end{enumerate}

\item \textbf{CreateHTML.java} \\
        An html file is created with the name \fch.

\item \textbf{RangeColor.java} \\
        This class provides static methods for computing an html color as a string from a hue value
        between 0 and 120. The various methods are overloaded. The methods share the name
        \textbf{htmlColor()}. For example \texttt{htmlColor(int color)} returns the String
        \texttt{"\#ff0000"} (pure red) on input 0, and \texttt{"\#00ff00"} (pure green) on input 120. For all other
        inputs in between 0 and 120, a linear blend of red and green is computed. For all invalid
        inputs, the string \texttt{"\#0000ff"} (pure blue) is returned to make sure that it is clear
        from the final report that an unexpected value was passed.
        input.

\item \fch\ \\
This file includes displays all the entire code with each statement highlighted by a color
corresponding to the value calculated by the above formula. Comments are highlighted with gray
color. Statements that are not executed by any run are not highlighted by any color. There is also a
slide bar locked at the top of the page which can be used to change the value of a threshold such
that the statements having their color value more than the threshold will be colored green and the
remaining statements will be recolored based on a new scale ranging from 0 to the threshold value.

\end{enumerate}

\section{Mutation Analysis}

\subsection{Python Source Mutation}
To validate the ability of \Uro\ to find bugs, we decided to implement mutation analysis on our
examples to more reliably sew errors into working examples. To achieve this, we created a module
that could be run from the command line that would take the path of a Python source file as an
argument and generate ``mutated" versions. These ``mutants'' would be identical to the original source,
with the exception that an operator from the original file would be changed to another operator that
could be found in the same place. The following operator ``classes'' were implemented:

\begin{itemize} 
\item Arithmetic operators: \texttt{(+, -, *, /, \%)} 
\item Augmented assignment
operators: (\texttt{+=, -=, *=, /=, \%=}) 
\item Numeric comparison operators: (\texttt{<, >, <=,
>=})
\item Equality comparison operators: \texttt{(==, !=)} 
\item Boolean comparison operators:
\texttt{(or, and)} 
\item Boolean literals: \texttt{(True, False)} 
\end{itemize}

The mutator module combs through a source file looking for any of these operators and enumerates all
possible mutants per operator. For example, the line

\texttt{x = y + z}

would create four mutant files by replacing the `+' operator with the other operators in its
``class,'' \texttt{(-, *, /, and \%).}

Should a line of code contain more than one instance of an operator (such as \texttt{x = y + z + w}), then
each ``instance" of the operator is treated as an independent mutation such that there is a total of
one mutation per file.

These mutated files are stored in a directory created by the module called, \texttt{mutants/}, located in the
same directory of the source file. Also included in the mutants directory is a ``mutant index,''
\texttt{mutants.txt}, that
details for each numbered mutant, respectively, the line number where the mutation occurs, the operator in the
original source file, the new
operator, and the instance number of the replaced operator in tab separated columns.

The mutator can be used to produce mutations on a file directly with the command: \\
\begin{verbatim}
$ mutator.py faultycode.py
\end{verbatim}

\subsection{Mutation Analysis Script}

In order to evaluate the \Taran\ approach to visualization (and \Uro's implementation of it), 
we have produced a script called
\textbf{mu\_analysis.sh}. The script is run with the source module and the test suite as arguments:
\begin{verbatim}
$ mu_analysis.sh faultycode.py faultycodeTest.py
\end{verbatim}
The script does the following:
\begin{enumerate}
\item Mutate the source file, producing a directory of mutants
\item For each mutant $m_i$, do the following
    \begin{enumerate}
    \item Run \Uro\ on the test suite, given $m_i$.
    \item Determine from the index, \textbf{mutants.txt}, which line was mutated in order to create
    $m_i$. Call this line $L_i$.
    \item Record both the average color \Uro\ assigns to all lines in $m_i$ except for $L_i$, and
    the color specifically assigned to $L_i$. Both of these values are written separated by a tab 
    to the file \textbf{faultycode\_colors.txt}.
    \end{enumerate}
\item Average both columns of \textbf{faultycode\_colors.txt}. The resulting two values are the
average color \Uro\ assigns to \emph{non-mutated} lines and the average color \Uro\ assigns to
\emph{mutated} lines. 
\end{enumerate}

For the two modules we tested that had non-trivial purposes and unit tests, the following
results were achieved: 

\begin{center}
\begin{tabular}{|l|l|l|}
    \hline
    \textbf{file} & \textbf{Average non-mutant hue} & \textbf{Average mutant hue} \\
    \hline
    BPT & 42.1751  & 12.4211 \\
    \hline
    negadecimal & 41.4801   &  32.0566 \\
    \hline
\end{tabular}
\end{center}
\textbf{Note:} Color values correspond to the 0 to 120 range \Uro\ uses to encode hue, where
0 is pure red, and 120 is pure green.



\bibliographystyle{plain}
\bibliography{BibEntries}
\end{document}
