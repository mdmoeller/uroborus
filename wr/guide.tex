\documentclass[english]{article}

\usepackage[latin9]{inputenc}
\usepackage[letterpaper]{geometry}
\geometry{verbose,tmargin=1in,bmargin=1in,lmargin=1in,rmargin=1in}
\usepackage{amsmath}
\usepackage{amssymb}
\usepackage{cite}

% Typeset UROBORUS:
\newcommand{\Uro}{\textsc{Uroborus}}
\newcommand{\Taran}{\textsc{Tarantula}}

% The `example' files referred to by this text:
\newcommand{\fc}{\texttt{faultycode}}
\newcommand{\fcp}{\texttt{faultycode.py}}
\newcommand{\fcT}{\texttt{faultycodeTest.py}}
\newcommand{\fcip}{\texttt{faultycode\_instrumented.py}}
\newcommand{\fcc}{\texttt{faultycode\_coverage.txt}}
\newcommand{\fcpf}{\texttt{faultycode\_passfail.txt}}
\newcommand{\fch}{\texttt{faultycode\_report.html}}



\title{Fault Localization with Uroborus}
\author{Samarth Kishore
    %%samarthk@seas.upenn.edu
\and 
Dan Klein
    %%kleindan@seas.upenn.edu
\and 
Mark Moeller
    %%mmoeller@seas.upenn.edu
\and 
Twisha Shah
    %%twisha@seas.upenn.edu
}



\begin{document}
\maketitle


\section{Introduction}

\Uro\ is a fault localization tool for Python built in the style of Jones, Harrold and Stasko's
\Taran ~\cite{Jones}.

Suppose we have written a Python module \fcp\ that we we want to debug with the
help of \Uro. At a high level, we need to do the following:
\begin{enumerate}
\item We instrument the input source file, \fcp.
\item We write a set of \Uro\ tests for \fc\ that call the methods of the instrumented source, in the
form of a script.
\item We run the test script, which generates datafiles detailing statement coverage and assertion
passes and failures.
\item We generate a `report,' which is an HTML file representing the source code, colored according
to the pass/failure calculations defined by \Taran.
\end{enumerate}
\textbf{Note:} At this time, \Uro\ is only able to test the \emph{functions} of an
input source file. That is, scripts are not handled under the current framework, but are an area being
considered for further improvement of the tool.

\section{Instrumenting a Source File}

To instrument a source file called \fcp, we just need to run the command:\\

\texttt{\$ instrument.py }\fcp\ \\

This will produce a file called \fcip\  which is the instrumented version of the source.

\section{Creating Test Modules}

A developer then needs to create \Uro-style tests for \fc. To do that, we just have to write a
script, say \fcT, that has the following header:\\
\begin{verbatim}
#!/usr/bin/python

import faultycode_instrumented as faultycode
target_module = faultycode

\end{verbatim}

From there, the test suite consists of \emph{methods} which are tests of the desired methods of \fc. 
That is, the script need not do anything when run directly. In particular, the module \emph{should
not} call any methods of instrumented code if run directly. 
Each method of the test suite should take one parameter, and that parameter is the
\texttt{RuntimeOracle} that will be used for that test case.

The developer may optionally include a method called \texttt{init} if some intial set up is
required. \Uro will run \texttt{init} if it exists before any other tests in the module.

Next, the module-level methods will each be
run in turn by \Uro, very similarly to how JUnit runs a set of methods. Order of execution of these
methods is not defined.
All assertions must be
done with \texttt{RuntimeOracle}'s assertion method (for details, see the documentation for RuntimeOracle).

\section{Localizing Faults}

Next, the developer runs the tests. This is done with:\\

\texttt{\$ uroborus.py }\fcT \\

When the tests are run, \Uro\
will generate \fcc\ and \fcpf, and it will write information to standard out about the tests that
pass and fail. Finally, we generate the visualization with:\\

\texttt{\$ java DisplayResult }\fc \\

This command will automatically find the datafiles it needs (assuming they have been properly
generated by \Uro, and not some other way) and it will output a file called \fch. This file is
the fault-visualized version of the source code.


\bibliographystyle{plain}
\bibliography{BibEntries}
\end{document}
